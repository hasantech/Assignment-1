\documentclass[a4paper,12pt]{article}
\usepackage[utf8]{inputenc}

\title{Google Developer Products:\\
\texttt{Google Maps}}
\author{LUKWAGO HASSAN NASSIL (16/U/6666/EVE, 216005795)}
\date{}
\begin{document}
\maketitle

\section{Introduction}
Google Maps is a web mapping service developed by Google. It offers satellite imagery, street maps,  panoramic views of streets (Street View), real-time traffic conditions (Google Traffic), and route planning for traveling by foot, car, bicycle (in beta), or public transportation.
s
Google Maps began as a C++ desktop program designed by Lars and Jens Eilstrup Rasmussen at Where 2 Technologies. In October 2004, the company was acquired by Google, which converted it into a web application. After additional acquisitions of a geospatial data visualization company and a realtime traffic analyzer, Google Maps was launched in February 2005.\cite{s1} The service's front end utilizes JavaScript, XML, and Ajax. Google Maps offers an API that allows maps to be embedded on third-party websites,\cite{s2} and offers a locator for urban businesses and other organizations in numerous countries around the world. Google Map Maker allowed users to collaboratively expand and update the service's mapping worldwide but was discontinued from March, 2017. However, crowdsourced contributions to Google Maps were not discontinued as the company announced those features will be transferred to Google Local Guides program.\cite{s3}

Google Maps' satellite view is a "top-down" or "birds eye" view; most of the high-resolution imagery of cities is aerial photography taken from aircraft flying at 800 to 1,500 feet (240 to 460 m), while most other imagery is from satellites.\cite{s4} Much of the available satellite imagery is no more than three years old and is updated on a regular basis.\cite{s5}Google Maps uses a close variant of the Mercator projection, and therefore cannot accurately show areas around the poles.



\section {Google maps services and claims}


\subsection {Google Moon}
\label{sec1}
In honor of the 36th anniversary of the Apollo 11 moon landing on July 20, 1969, Google took public domain imagery of the Moon, integrated it into the Google Maps interface, and created a tool called Google Moon.\cite{s6} By default this tool, with a reduced set of features, also displays the points of landing of all Apollo spacecraft to land on the Moon. It also included an Easter egg, displaying a Swiss cheese design at the highest zoom level, which Google has since removed.[citation needed] A collaborative project between NASA Ames Research Center and Google called the Planetary Content Project integrates and improves the data that is used for Google Moon.\cite{s7} Google Moon was linked from a special commemorative version of the Google logo displayed at the top of the main Google search page for July 20, 2005 (UTC).\cite{s8}

\subsection{Google Mars}
Google Mars provides a visible imagery view, like Google Moon, as well as infrared imagery and shaded relief (elevation) of the planet Mars. Users can toggle between the elevation, visible, and infrared data, in the same manner as switching between map, satellite, and hybrid modes of Google Maps. In collaboration with NASA scientists at the Mars Space Flight Facility located at Arizona State University, Google has provided the public with data collected from two NASA Mars missions, Mars Global Surveyor and 2001 Mars Odyssey.\cite{s9}

Now, with Google Earth 5 it is possible to access new improved Google Mars data at a much higher resolution, as well as being able to view the terrain in 3D, and viewing panoramas from various Mars landers in a similar way to Google Street View.




\bibliographystyle{IEEEtran}
\bibliography{reference2}

\end{document}


